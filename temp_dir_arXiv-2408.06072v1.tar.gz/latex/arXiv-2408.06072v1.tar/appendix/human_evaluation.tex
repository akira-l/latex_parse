\clearpage
\section{Human Evaluation Details}\label{sec:human_evalution}

Sensory Quality: This part focuses mainly on the perceptual quality of videos, including subject consistency, frame continuity, and stability.

\begin{table}[h]
\centering
\caption{Sensory Quality Evaluation Criteria.}
\label{sample-table}
\small

\begin{tabular}{cp{11cm}}
\toprule

\textbf{Score} & \textbf{Evaluation Criteria} \\
\midrule
1  & High sensory quality: 1. The appearance and morphological features of objects in the video are completely consistent 2. High picture stability, maintaining high resolution consistently 3. Overall composition/color/boundaries match reality 4. The picture is visually appealing \\
\midrule
0.5  & Average sensory quality: 1. The appearance and morphological features of objects in the video are at least 80\% consistent 2. Moderate picture stability, with only 50\% of the frames maintaining high resolution 3. Overall composition/color/boundaries match reality by at least 70\% 4. The picture has some visual appeal \\
\midrule
0  & Poor sensory quality: large inconsistencies in appearance and morphology, low video resolution, and composition/layout not matching reality \\

\bottomrule
\end{tabular}
\end{table}

Instruction Following: This part focuses on whether the generated video aligns with the prompt, including the accuracy of the subject, quantity, elements, and details.

\begin{table}[h]
\centering
\caption{Instruction Following Evaluation Criteria.}
\label{sample-table}
\small

\begin{tabular}{cp{11cm}}
\toprule

\textbf{Score} & \textbf{Evaluation Criteria} \\
\midrule
1  & 100\% follow the text instruction requirements, including but not limited to: elements completely correct, quantity requirements consistent, elements complete, features accurate, etc. \\
\midrule
0.5  & 100\% follow the text instruction requirements, but the implementation has minor flaws such as distorted main subjects or inaccurate features. \\
\midrule
0  & Does not 100\% follow the text instruction requirements, with any of the following issues:  1. Generated elements are inaccurate  2. Quantity is incorrect  3. Elements are incomplete  4. Features are inaccurate \\
\bottomrule
\end{tabular}
\end{table}



Physics Simulation: This part focuses on whether the model can adhere to the objective law of the physical world, such as the lighting effect, interactions between different objects, and the realism of fluid dynamics. 


\begin{table}[h]
\centering
\caption{Physics Simulation Evaluation Criteria.}
\label{sample-table}
\small

\begin{tabular}{cp{11cm}}
\toprule

\textbf{Score} & \textbf{Evaluation Criteria} \\
\midrule
1  & Good physical realism simulation capability, can achieve: 1. Real-time tracking 2. Good action understanding, ensuring dynamic realism of entities 3. Realistic lighting and shadow effects, high interaction fidelity 4. Accurate simulation of fluid motion \\
\midrule
0.5  & Average physical realism simulation capability, with some degradation in real-time tracking, dynamic realism, lighting and shadow effects, and fluid motion simulation. Issues include: 1. Slightly unnatural transitions in dynamic effects, with some discontinuities 2. Lighting and shadow effects not matching reality 3. Distorted interactions between objects 4. Floating fluid motion, not matching reality \\
\midrule
0  & Poor physical realism simulation capability, results do not match reality, obviously fake \\

\bottomrule
\end{tabular}
\end{table}



\newpage
Cover Quality: This part mainly focuses on metrics that can be assessed from single-frame images, including aesthetic quality, clarity, and fidelity.

\begin{table}[h]
\centering
\caption{Cover Quality Evaluation Criteria.}
\label{sample-table}
\small

\begin{tabular}{cp{11cm}}
\toprule

\textbf{Score} & \textbf{Evaluation Criteria} \\
\midrule
1 & Image is clear, subject is obvious, display is complete, color tone is normal. \\
\midrule
0.5 & Image quality is average. The subject is relatively complete, color tone is normal. \\
\midrule
0 & Cover image resolution is low, image is blurry. \\

\bottomrule
\end{tabular}
\end{table}
