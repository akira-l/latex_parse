\section{Introduction}

In recent years, diffusion models have made groundbreaking advancements in multimodal generation, such as image, video, speech and 3D generation. Among these, video generation is a rapidly evolving field and being extensively explored. Given the successful experiences with Large Language Models (LLMs), comprehensive scaling up of data volume, training iterations, and model size consistently enhances model performance. Additionally, there is more mature scaling experience with transformers compared to UNet. And DiT \citep{peebles2023scalable} has shown that transformers can effectively replace UNet as the backbone of diffusion models. Thus, transformer is a better choice for video generation.
However, long-term consistent video generation remains a significant challenge.  % 第一段介绍任务重要性和我们工作的亮点,例如开源。DiT的讨论不应该在这里

The first challenge is that constructing a web-scale video data pipeline is considerably more difficult than for textual data. Video data is extremely diverse in distribution, quality varies greatly, and simple rule-based filtering is often insufficient for effective data selection. Consequently, processing video data is both time-consuming and highly complex.
There are numerous meaningless unrealistic videos, such as poor-quality edits and computer screen recordings. And many videos are difficult to watch normally, such as those with excessively shaky cameras. These types of data are harmful to the generative model's ability to learn genuine dynamic information. They need to be meticulously processed and filtered out to ensure the quality of the training dataset.

Additionally, most video data available online lacks accurate textual descriptions, significantly limiting the model's ability to grasp precise semantic understanding. To address this issue, we trained a video understanding model capable of accurately describing video content. We use it to generate new textual descriptions for all video data. 
To advance the field of video generation, we have decided to open-source this description model.

The high training cost is another significant challenge. If the video is unfolded into a one-dimensional sequence in the pixel space, the length would be extraordinarily long. To keep the computational cost within a feasible range, we trained a 3D VAE that compresses the video along both spatial and temporal dimensions. Additionally,  unlike previous video models that use a 2D VAE to encode each frame separately, 3D VAE ensures continuity among frames so that the generated videos do not flicker. 

Moreover, to improve the alignment between videos and texts, we propose an expert transformer to facilitate the interaction between the two modalities. Then, to ensure the consistency of video generation and to capture large-scale motions, it is necessary to comprehensively model the video along both temporal and spatial dimensions. Therefore, we opt for 3D full attention, as detailed in Section~\ref{sec:expert-transformer}.


