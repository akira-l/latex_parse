\section{Empirical Evaluation}
We trained a series of models of various sizes. For all subsequent evaluations, we will use the largest model (referred to as CogVideoX).
In this section, we present the experimental validation of CogVideoX through two primary methods: automated metric evaluation and human assessment, providing a thorough analysis of the performance and quality of the generated videos. 
We trained a series of models with different parameter sizes. The following evaluation defaults to using our largest model.

\subsection{Results of Automated Metric Evaluation} 

\paragraph{Baselines.} We chose several top-performing text-to-video models as our baselines for comparison, including T2V-Turbo~\citep{li2024t2v}, AnimateDiff~\citep{guo2023animatediff}, VideoCrafter2~\citep{chen2024videocrafter2}, OpenSora~\citep{opensora}, Show-1~\citep{zhang2023show}, Gen-2~\citep{gen2}, Pika~\citep{pika} and LaVie-2~\citep{wang2023lavie}.


% \begin{figure}[h]
% \begin{center}
% \includegraphics[width=0.9\linewidth]{images/bench_eval.png}
% \end{center}
% \caption{The radar chart comparing the performance of different models.}
% \label{fig:radar}
% \end{figure}

\hide{
%\begin{wrapfigure}{r}{0.5\textwidth}
\begin{figure}
\centering
\includegraphics[width=0.7\linewidth]{images/bench_eval9.png}
\caption{The radar chart comparing the performance of different models. CogVideoX represents the largest one. It is clear that CogVideoX outperforms its competitors in the vast majority of metrics, and it is very close to the leading models in the remaining indicator.
}
\label{fig:radar}
% \vspace{-10mm}
%\end{wrapfigure}

\end{figure}

}%end ofhide
\paragraph{Evaluation Metrics.} To evaluate the text-to-video generation, we employed several metrics from VBench~\citep{huang2023vbench}: \emph{Human Action}, \emph{Scene}, \emph{Dynamic Degree}, \emph{Multiple Objects}, and \emph{Appearance Style}. VBench is a suite of tools designed to automatically assess the quality of generated videos. We have selected certain metrics from VBench, excluding others that do not align with our evaluation needs. For example, the color metric, intended to measure the presence of objects corresponding to specific colors across frames in the generated video, assesses the model's quality by calculating the probability. However, this metric may mislead video generation models that exhibit greater variation, thus we chose not to include it in our evaluation. For longer-generated videos, some models might produce videos with minimal changes between frames to obtain higher scores, but these videos lack rich content. Therefore, a metric for evaluating the dynamism of the video becomes more important. To address this, we employed two video evaluation tools, We also employed the \emph{Dynamic Quality} from Devil~\citep{liao2024evaluationtexttovideogenerationmodels} and \emph{GPT4o-MTScore} from ChronoMagic~\citep{yuan2024chronomagic}, which focus more on the dynamic characteristics of videos. \emph{Dynamic Quality} is defined by the integration of various quality metrics with dynamic scores. This approach mitigates biases arising from negative correlations between video dynamics and video quality, leading to a more thorough assessment of video quality. ChronoMagic, for instance, introduces the \emph{GPT4o-MTScore}, a metric designed to measure the metamorphic amplitude of time-lapse videos, such as those depicting physical, biological, and meteorological changes. This metric is obtained by extracting frames from the generated videos at regular intervals and using GPT-4o~\citep{gpt4o} to score the degree of change, providing a fine-grained assessment of video dynamism. This method ensures a more accurate evaluation of the content's variability over time, countering the potential bias of static frame sequences in scoring.



\paragraph{Results.} Table~\ref{table:results} provides a detailed comparison of the performance of our CogVideoX model with other models. Our model achieved the best performance in 5 out of the 7 metrics and showed competitive results in the remaining 2 metrics. These results demonstrate that our model not only excels in video generation quality but also outperforms previous models in handling various complex dynamic scenes. Additionally, Figure~\ref{fig:radar} presents a radar chart comparing the performance of different models.


\begin{figure}[ht]
\begin{center}
\includegraphics[width=\linewidth]{images/t2v/goodcase1.jpg}
\end{center}
\caption{Text to video showcases. The displayed prompt will be upsampled before being fed into the model. The generated videos contain large motion and can produce various video styles.}
\label{fig:t2vgood1}
\end{figure}

\begin{figure}[ht]
\begin{center}
\includegraphics[width=0.98\linewidth]{images/t2v/goodcase2.jpg}
\end{center}
\caption{Text to video showcases.}
\label{fig:t2vgood2}
\end{figure}


% Please add the following required packages to your document preamble:
% \usepackage[table,xcdraw]{xcolor}
% Beamer presentation requires \usepackage{colortbl} instead of \usepackage[table,xcdraw]{xcolor}
% \usepackage[normalem]{ulem}
% \useunder{\uline}{\ul}{}




% \begin{table}[]

% \centering
% \setlength\tabcolsep{3pt}

% \label{sample-table}
% \small
% \vspace{-10pt}
% \caption{\textbf{Automatic Evaluation Results per Dimension.}The table presents a comparative analysis of various video models across different dimensions. It is evident from the table that, in terms of both human motion and background effects as well as the accuracy and distinctiveness of objects, CogVideoX has achieved the current SOTA level. Furthermore, CogVideoX has garnered a commendable score in the expression of dynamic qualities, a capability that serves as a more precise indicator of the intrinsic properties of video media, distinct from the static nature of photographic images.}

% \vspace{6pt}

% \begin{tabular}{cccccccc}
% \toprule
% \multirow{2}{*}{\textbf{Models} }  & \textbf{human}  & \textbf{object} &\multirow{2}{*}{\textbf{scene}}&\textbf{dynamic} &\textbf{multiple} &\textbf{spatial} &\textbf{appearance} \\
%     & \textbf{action}& \textbf{class}& & \textbf{degree} &\textbf{objects}& \textbf{relationship}&\textbf{style}  
% \\
% \midrule
% CogVideoX & 96.80\% &93.70\% & 55.44\% & 62.22\% & 70.95\% & 61.29\% & 24.44\% \\
% {LaVie-2} & 96.40\% & 97.52\%  & 49.59\% & 31.11\% & 64.88\%  & 38.68\% & 25.09\%  \\
% {T2V-Turbo}  & 95.20\%  & 93.96\%& 55.58\% & 49.17\% & 54.65\%    & 38.67\%  & 24.42\%   \\
% {Gen-2}  & 89.20\%& 90.92\%  & 48.91\%  & 18.89\% & 55.47\%    & 66.91\%   & 19.34\%  \\
% {VideoCrafter-2.0\citep{chen2024videocrafter2}} & 95.00\% & 92.55\% & 55.29\%               & 42.50\% & 40.66\% & 35.86\% & 25.13\%  \\
% {Pika Beta} & 88.00\% & 87.45\%  & 44.80\% & 37.22\% & 46.69\% & 65.65\% & 21.89\%   \\
% AnimateDiff-V2 & 92.60\% & 90.90\%  & 50.19\% & 40.83\%        & 36.88\% & 34.60\%  & 22.42\%\\
% {OpenSora V1.2}   & 85.80\% & 83.37\%& 42.47\%   & 47.22\%    & 58.41\% & 67.51\%  & 23.89\%  \\
% {Show-1} & 95.60\%  & 93.07\%  & 47.03\% & 44.44\% & 45.47\% & 53.50\%  & 23.06\%  \\
% {HiGen}  & 86.20\%  & 86.06\%  & 44.88\% & 99.17\% & 22.39\%  & 22.43\% & 24.54\% \\  
% \bottomrule
% \end{tabular}
% \end{table}



% \iffalse



% \begin{table}[ht!]
% \centering
% \caption{Evaluation results.}
% \setlength\tabcolsep{3pt}
% \label{sample-table}
% \begin{center}
% \small
% \resizebox{0.9\linewidth}{!}{
% \begin{tabular}{ccccccccc}

% \multirow{2}{*}{\textbf{Models} }  & \textbf{subject}  & \textbf{background} &\textbf{temporal} &\textbf{motion} &\textbf{dynamic} &\textbf{aesthetic} &\textbf{imaging} &\textbf{object} \\
%     & \textbf{consistency}& \textbf{consistency}& \textbf{flickering}& \textbf{smoothness} &\textbf{degree}& \textbf{quality}&\textbf{quality} & \textbf{class}
% \\ \hline 
%         CogVideoX & 94.66\% & 95.92\% & 97.47\% & 98.10\% & 62.22\% & 55.14\% & 63.62\% & 93.70\%  \\
%         LaVie-2 & 97.90\% & 98.45\% & 98.76\% & 98.42\% & 31.11\% & 67.62\% & 70.39\% & 97.52\%  \\ 
%         T2V-Turbo (VC2) & 96.28\% & 97.02\% & 97.48\% & 97.34\% & 49.17\% & 63.04\% & 72.49\% & 93.96\%  \\ 
%         Gen-2 (2023-06) & 97.61\% & 97.61\% & 99.56\% & 99.58\% & 18.89\% & 66.96\% & 67.42\% & 90.92\%  \\ 
%         VideoCrafter-2.0\citep{chen2024videocrafter2} & 96.85\% & 98.22\% & 98.41\% & 97.73\% & 42.50\% & 63.13\% & 67.22\% & 92.55\%  \\ 
%         Pika Beta (2023-06) & 96.76\% & 98.95\% & 99.77\% & 99.51\% & 37.22\% & 63.15\% & 62.33\% & 87.45\%  \\ 
%         AnimateDiff-V2 & 95.30\% & 97.68\% & 98.75\% & 97.76\% & 40.83\% & 67.16\% & 70.10\% & 90.90\%  \\ 
%         OpenSora V1.2 & 94.45\% & 97.90\% & 99.47\% & 98.20\% & 47.22\% & 56.18\% & 60.94\% & 83.37\%  \\ 
%         Show-1 & 95.53\% & 98.02\% & 99.12\% & 98.24\% & 44.44\% & 57.35\% & 58.66\% & 93.07\%  \\ 
%         HiGen & 90.07\% & 93.99\% & 93.24\% & 96.69\% & 99.17\% & 57.30\% & 63.92\% & 86.06\% \\ 
% \hline \\

% \multirow{2}{*}{\textbf{Models} }  & \textbf{multiple}  & \textbf{human} &\multirow{2}{*}{\textbf{color}} &\textbf{spatial} &\multirow{2}{*}{\textbf{scene}} &\textbf{appearance} &\textbf{temporal} &\textbf{overall} \\
%     & \textbf{objects}& \textbf{action}& & \textbf{relation} & & \textbf{style}&\textbf{style} & \textbf{consistency}
% \\ \hline 
%         CogVideoX & 70.95\% & 96.80\% & 79.75\% & 61.29\% & 55.44\% & 24.44\% & 23.69\% & 26.73\%  \\ 
%         LaVie-2 & 64.88\% & 96.40\% & 91.65\% & 38.68\% & 49.59\% & 25.09\% & 25.24\% & 27.39\%  \\ 
%         T2V-Turbo (VC2) & 54.65\% & 95.20\% & 89.90\% & 38.67\% & 55.58\% & 24.42\% & 25.51\% & 28.16\%  \\
%         Gen-2 (2023-06) & 55.47\% & 89.20\% & 89.49\% & 66.91\% & 48.91\% & 19.34\% & 24.12\% & 26.17\%  \\ 
%         VideoCrafter-2.0 & 40.66\% & 95.00\% & 92.92\% & 35.86\% & 55.29\% & 25.13\% & 25.84\% & 28.23\%  \\
%         Pika Beta (2023-06) & 46.69\% & 88.00\% & 85.31\% & 65.65\% & 44.80\% & 21.89\% & 24.44\% & 25.47\%  \\ 
%         AnimateDiff-V2 & 36.88\% & 92.60\% & 87.47\% & 34.60\% & 50.19\% & 22.42\% & 26.03\% & 27.04\%  \\ 
%         OpenSora V1.2 & 58.41\% & 85.80\% & 87.49\% & 67.51\% & 42.47\% & 23.89\% & 24.55\% & 27.07\%  \\ 
%         Show-1 & 45.47\% & 95.60\% & 86.35\% & 53.50\% & 47.03\% & 23.06\% & 25.28\% & 27.46\%  \\ 
%         HiGen & 22.39\% & 86.20\% & 86.22\% & 22.43\% & 44.88\% & 24.54\% & 25.14\% & 27.14\% \\ \hline

% \hline \\
% \end{tabular}

% }
% \end{center}
% \end{table}

% \fi






% \begin{table}[!ht]
% \centering

% \label{sample-table}
% \small
% \vspace{-10pt}
% \caption{\textbf{Automatic Evaluation Results per Dimension.}}

% \vspace{6pt}

% \resizebox{0.8\linewidth}{!}{
%     \begin{tabular}{cccc}
%         \textbf{Models} & \textbf{\Centerstack{Dynamics Range}} & \textbf{\Centerstack{Dynamics Controllability}} & \textbf{\Centerstack{Dynamics-based Quality}} \\ \hline
%         CogVideoX       & 55.7 & 71.8 & \textbf{69.5} \\ 
%         Gen-2           & 30.8 & \textbf{82.5} & 43.6 \\ 
%         Pika            & 43.2 & 72.0 & 52.1 \\ 
%         VideoCrafter2   & 34.1 & 57.0 & 43.6 \\ 
%         OpenSora        & \textbf{61.2} & 62.4 & 63.7 \\ 
%         Show-1          & 45.1 & 73.9 & 57.7 \\ 
%     \end{tabular}
% }
% \end{table}


% \begin{figure}[h]
% \begin{center}
% \includegraphics[width=0.9\linewidth]{images/bench_eval.png}
% \end{center}
% \caption{The radar chart comparing the performance of different models.}
% \label{fig:radar}
% \end{figure}

\hide{
%\begin{wrapfigure}{r}{0.5\textwidth}
\begin{figure}
\centering
\includegraphics[width=0.7\linewidth]{images/bench_eval9.png}
\caption{The radar chart comparing the performance of different models. CogVideoX represents the largest one. It is clear that CogVideoX outperforms its competitors in the vast majority of metrics, and it is very close to the leading models in the remaining indicator.
}
\label{fig:radar}
% \vspace{-10mm}
%\end{wrapfigure}

\end{figure}

}%end ofhide


\subsection{Human Evaluation}
In addition to automated scoring mechanisms, a comparative analysis between the Kling~\citep{kling} and CogVideoX was conducted using a manual scoring system. One hundred meticulously crafted prompts were used, characterized by their broad distribution, clear articulation, and well-defined conceptual scope. We randomize videos for blind evalution. A panel of evaluators assigned scores for each detail on a scale from zero to one, with the overall total score rated on a scale from zero to five, where higher scores reflect better video quality. Reasons for any score deductions were also carefully documented. The results shown in Table~\ref{table:human_eva} indicate that our model outperforms Kling in all aspects. More details are shown in \ref{sec:human_evalution}.

\begin{table}[!ht]
\centering
\label{sample-table}
\small
\vspace{-5pt}
\caption{Human evaluation between CogVideoX and Kling.}
\label{table:human_eva}
\resizebox{0.75\linewidth}{!}{
    \begin{tabular}{cccccc}
    \toprule
        Model & \Centerstack{Sensory\\Quality} & \Centerstack{Instruction\\Following}&\Centerstack{Physics\\Simulation} & \Centerstack{Cover\\Quality} & 
        \Centerstack{Total\\Score} \\ 
        \midrule
        Kling & 0.638 & 0.367 & 0.561 & 0.668 & 2.17 \\
        \midrule
         {\bf CogVideoX-5B} & {\bf 0.722} & {\bf 0.495} & {\bf 0.667} & {\bf 0.712} & {\bf 2.74}  \\
        \bottomrule
    \end{tabular}
}
\end{table}



% \begin{table}[!ht]
% \centering

% \label{sample-table}
% \small
% \vspace{-10pt}
% \caption{\textbf{Automatic Evaluation Results per Dimension.}}

% \vspace{6pt}

% \resizebox{0.8\linewidth}{!}{
%     \begin{tabular}{cccc}
%         \textbf{Models} & \textbf{\Centerstack{Dynamics Range}} & \textbf{\Centerstack{Dynamics Controllability}} & \textbf{\Centerstack{Dynamics-based Quality}} \\ \hline
%         CogVideoX       & 55.7 & 71.8 & \textbf{69.5} \\ 
%         Gen-2           & 30.8 & \textbf{82.5} & 43.6 \\ 
%         Pika            & 43.2 & 72.0 & 52.1 \\ 
%         VideoCrafter2   & 34.1 & 57.0 & 43.6 \\ 
%         OpenSora        & \textbf{61.2} & 62.4 & 63.7 \\ 
%         Show-1          & 45.1 & 73.9 & 57.7 \\ 
%     \end{tabular}
% }
% \end{table}
